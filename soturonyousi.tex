\documentclass[10pt,a4paper,twocolumn]{jarticle}
\usepackage{geometry}
\usepackage{pxchfon}
\usepackage{titlesec}
\usepackage{multicol}
\geometry{left=20mm,right=20mm,top=20mm,bottom=20mm}
\geometry{textwidth=40zw,lines=40}
\titleformat*{\section}{\normalsize}
\titleformat*{\subsection}{\normalsize}
\titleformat*{\subsubsection}{\normalsize}
\begin{document}
\twocolumn[
\begin{center}
{\large 卒業論文発表 要旨(2019年度版)\\
ワイヤレス給電システムの最適化のための能率的な動作周波数スイープ\\}
\end{center}
\begin{flushright}
	自動制御研究室\\
	森田 光流 \\
\end{flushright}
]
\section{緒言}
\subsection{背景}
ワイヤレス給電とはコネクタや金属接点の接触を用いず無線で電力を供給・伝搬することが可能な給電方法である.金属接点がない代わりに送電・受電にコイルを用いることにより,送電側の交流電源の電流が時間的に変化することにより受電側の磁束が変化しファラデーの電磁誘導の法則ににより誘導起電力が生じ誘導電流を伝えるという原理である.従来の電気製品の金属接点やコネクタを使用したものは水がかかると水による感電やショートをおこす,配線による転倒などの安全性に関して問題点がある.しかしワイヤレス給電は金属接点がないため前述の問題点を解決することができる.また非金属のものであればコイル間に存在していても送電側と受電側の電力に影響を及ぼさないため,人が立ち寄れない危険な場所や人体の中などにある機器や装置の遠隔操作ができるという点がある.\\ また近年の研究では地球温暖化の原因の一つである二酸化炭素の削減のためには電気自動車の開発並びに普及が望まれる.現在電気自動車に使用されている接触式充電方式では雨天時でプラグを扱うと感電する恐れがある.それをワイヤレス給電置き換えると自宅の駐車場に駐車するだけで充電が可能になり感電の可能性が払しょくされ充電の際に運転者が降車する必要がなくなり運転者の負担が大きく軽減されることが期待される.
\subsection{目的}
\section{実験内容}
\section{実験結果・考察}
\section{今後の方針}

\end{document}
